% Created 2017-05-01 Mon 18:23
\documentclass[presentation]{beamer}
\usepackage[utf8]{inputenc}
\usepackage[T1]{fontenc}
\usepackage{fixltx2e}
\usepackage{graphicx}
\usepackage{longtable}
\usepackage{float}
\usepackage{wrapfig}
\usepackage{rotating}
\usepackage[normalem]{ulem}
\usepackage{amsmath}
\usepackage{textcomp}
\usepackage{marvosym}
\usepackage{wasysym}
\usepackage{amssymb}
\usepackage{hyperref}
\tolerance=1000
\usepackage[utf8]{inputenc}
\usetheme{metropolis}
\author{Oleg Sivokon}
\date{\textit{<2017-05-01 Mon>}}
\title{Using Golog (Prolog interpretator in Go)}
\hypersetup{
  pdfkeywords={go golang prolog golog interpretator},
  pdfsubject={Overview and Prolog interpretator for Go},
  pdfcreator={Emacs 25.2.1 (Org mode 8.2.10)}}
\begin{document}

\maketitle
\begin{frame}{Outline}
\tableofcontents
\end{frame}


\section{But Why?}
\label{sec-1}
\includegraphics[height=8cm]{./images/butwhy.jpeg}

\begin{frame}[label=sec-1-1]{Serialization}
\begin{quote}
This comes up when you're doing generic programming, where you
want to write functions that will take data of any type and just
walk over it and serialize it, say.  So that's a time when types
can be a bit awkward and an untyped language is particularly
straightforward.  It couldn't be easier to write a serializer in
an untyped language.
\end{quote}
Simon Peyton Jones, GHC author, Microsoft Research.
\end{frame}

\begin{frame}[label=sec-1-2]{Parsing}
\begin{quote}
Prolog was originally invented as a programming language in which
to write natural language applications, and thus Prolog is a very
elegant language for expressing grammars.  Prolog even has a
builtin syntax especially created for writing grammars.  It is
often said that with Prolog one gets a builtin parser for free.
\end{quote}
Prof. David S. Warren, University of New York
\end{frame}

\begin{frame}[label=sec-1-3]{Homoiconicity}
\begin{quote}
The ability to represent the code in the language by other code in
that language makes that language deserve the status of
homo-iconic.  Since version 3.0 C\# was homoiconic for the
expression subset of the language, thanks to the introduction of
expression trees.  As you can see, the expression trees are a very
powerful concept to inspect code at runtime which goes much beyond
their use in LINQ.  You can think of them as ``reflection on
steroids'', and their importance will only increase in the
releases to come.
\end{quote}
Bart De Smet, C\# 5.0.
\end{frame}

\begin{frame}[label=sec-1-4]{Larger problem: dealing with the unknown}
\end{frame}

\begin{frame}[label=sec-1-5]{Some practical uses}
\begin{itemize}
\item Knowledge bases
\item Interactive shells
\item Rich data exchange and configuration format
\item Testing
\end{itemize}
\end{frame}

\section{Water, From The Toilet?}
\label{sec-2}
\includegraphics[height=8cm]{./images/water-from-toilet.jpg}

\begin{frame}[label=sec-2-1]{Various models of computation, unknown to general public}
\begin{itemize}
\item Petri nets a.k.a. Turing machine
\item ``Value semantics'', a.k.a. ``functional programming''
a.k.a. ``declarative programming''
\item Unification
\item Graph rewriting
\item Cellular automata
\item Biological modes? Perhaps much, much more\ldots{}
\end{itemize}
\end{frame}

\begin{frame}[label=sec-2-2]{Programming taught using one model}
\begin{itemize}
\item Model intuitively accepted as the only possible one
\item Students don't develop more abstract understanding of computation
\item Some problems lend themselvs better to particular computation
model
\end{itemize}
\end{frame}

\begin{frame}[fragile,label=sec-2-3]{Unification}
 \begin{verbatim}
%% Two uninstantiated variables unify
X = Y,
%% Unification works on sub-terms
%% X = 1
[X | Xs] = [1 | Xs],
%% Unification can be arbitrary deep
%% and unify multiple variables
%% X = 1, Y = 2, Z = baz(3)
foo([bar(X, Y), Z]) = foo([bar(1, 2), baz(3)]),
%% Unification can be recursive
%% X = rec(rec(rec(...)))
X = rec(X)
\end{verbatim}
\end{frame}

\section{Seriously, Now}
\label{sec-3}
\includegraphics[height=6cm]{./images/serious.jpg}

\begin{frame}[label=sec-3-1]{Obtaining the code}
\begin{itemize}
\item Official version: git@github.com:mndrix/golog.git
\item My fork: git@github.com:wvxvw/golog.git
\end{itemize}
\end{frame}

\begin{frame}[fragile,label=sec-3-2]{Minimal setup}
 \begin{verbatim}
m := golog.NewInteractiveMachine()
m = m.Consult(`
    father(john).
    father(jacob).
    mother(sue).
    parent(X) :- father(X).
    parent(X) :- mother(X).
`)
solutions := m.ProveAll(`parent(X).`)
for _, s := range solutions {
    fmt.Printf("%s is a parent\n", s.ByName_("X"))
}
\end{verbatim}
\end{frame}

\begin{frame}[label=sec-3-3]{golog.Machine}
\begin{itemize}
\item Main interface to interpreter
\item Immutable \emph{(safe for concurrent access)}
\end{itemize}
\end{frame}

\begin{frame}[fragile,label=sec-3-4]{(*golog.Machine).Consult(interface\{\}) *golog.Machine}
 \begin{itemize}
\item \texttt{consult/1} loads code in ISO Prolog
\item Creates database
\item Similar to \texttt{insert} in SQL
\end{itemize}
\end{frame}

\begin{frame}[fragile,label=sec-3-5]{(*golog.Machine).ProveAll(interface\{\}) []golog.Bindings}
 \begin{itemize}
\item Generates \alert{ALL} solutions
\item Similar to \texttt{select} in SQL
\end{itemize}
\end{frame}

\begin{frame}[fragile,label=sec-3-6]{Exposing foreign (native) objects}
 \begin{verbatim}
func HelloWorld(
    m golog.Machine, args []term.Term,
) golog.ForeignReturn {
    return golog.ForeignUnify(
        term.NewAtom("Hello World"), args[0],
    )
}
func NewHelloWorldMachine() golog.Machine {
    return golog.NewMachine().RegisterForeign(
        map[string]golog.ForeignPredicate{
            "hello_world/1": HelloWorld,
        },
    )
}
\end{verbatim}
\end{frame}

\begin{frame}[fragile,label=sec-3-7]{Using foreign objects}
 \begin{verbatim}
m := golog.NewHelloWorldMachine()
solutions := m.ProveAll(`hello_world(X).`)
for _, s := range solutions {
    fmt.Printf("%s\n", s.ByName_("X")) // Hello World
}
\end{verbatim}
\end{frame}

\begin{frame}[fragile,label=sec-3-8]{Marshalling Go structs (experimental)}
 \begin{itemize}
\item \texttt{native.Decoder} decodes Prolog callable terms into Go structs
\item \texttt{native.Encoder} encodes Go structs as Prolog terms
\item \texttt{native.GenerateMethods()} generates methods specializing on
given struct
\item \texttt{native.GenerateAccessors()} generates special Prolog predicates
for accessing fields in Go structs
\item \alert{Limitations:} Go methods don't backtrack.  Some reasonable
functionality is still to be implemented.
\end{itemize}
\end{frame}

\begin{frame}[fragile,label=sec-3-9]{Extending}
 \begin{itemize}
\item Example: \texttt{term.Native} type
\item \texttt{golog.RegisterForeign} to create machines with special set of
predicates
\item \texttt{golog.Machine} is an \emph{interface!} Possible other implementations
include:
\begin{itemize}
\item Relational database access
\item Distributed machines
\item Concurrent machines
\end{itemize}
\end{itemize}
\end{frame}
% Emacs 25.2.1 (Org mode 8.2.10)
\end{document}
